\RequirePackage[hyphens]{url}
\documentclass[sigchi]{acmart}
%% Include the standard set of goodies
\include{builtins/std}

%%
%% \BibTeX command to typeset BibTeX logo in the docs
\AtBeginDocument{%
  \providecommand\BibTeX{{%
    \normalfont B\kern-0.5em{\scshape i\kern-0.25em b}\kern-0.8em\TeX}}}

%% Rights management information.  This information is sent to you
%% when you complete the rights form.  These commands have SAMPLE
%% values in them; it is your responsibility as an author to replace
%% the commands and values with those provided to you when you
%% complete the rights form.
% \setcopyright{acmcopyright}
% \copyrightyear{2018}
% \acmYear{2018}
% \acmDOI{10.1145/1122445.1122456}

%% These commands are for a PROCEEDINGS abstract or paper.
% \acmConference[Woodstock '18]{Woodstock '18: ACM Symposium on Neural
%   Gaze Detection}{June 03--05, 2018}{Woodstock, NY}
% \acmBooktitle{Woodstock '18: ACM Symposium on Neural Gaze Detection,
%   June 03--05, 2018, Woodstock, NY}
% \acmPrice{15.00}
% \acmISBN{978-1-4503-XXXX-X/18/06}


%%
%% Submission ID.
%% Use this when submitting an article to a sponsored event. You'll
%% receive a unique submission ID from the organizers
%% of the event, and this ID should be used as the parameter to this command.
%%\acmSubmissionID{123-A56-BU3}

%%
%% The majority of ACM publications use numbered citations and
%% references.  The command \citestyle{authoryear} switches to the
%% "author year" style.
%%
%% If you are preparing content for an event
%% sponsored by ACM SIGGRAPH, you must use the "author year" style of
%% citations and references.
%% Uncommenting
%% the next command will enable that style.
%%\citestyle{acmauthoryear}

%%
%% end of the preamble, start of the body of the document source.
\begin{document}

%%
%% The "title" command has an optional parameter,
%% allowing the author to define a "short title" to be used in page headers.
\title{The Name of the Title is Hope}

\author{Michael James}
\affiliation{%
  \institution{University of California, San Diego}
}
\email{m3james@eng.ucsd.edu}


\author{John Renner}
\affiliation{%
  \institution{University of California, San Diego}
}
\email{jmrenner@eng.ucsd.edu}

%%
%% By default, the full list of authors will be used in the page
%% headers. Often, this list is too long, and will overlap
%% other information printed in the page headers. This command allows
%% the author to define a more concise list
%% of authors' names for this purpose.
\renewcommand{\shortauthors}{Trovato and Tobin, et al.}

%%
%% The abstract is a short summary of the work to be presented in the
%% article.
\begin{abstract}
Program synthesis takes the hard work out of \emph{writing} code, at the cost of
having to \emph{read} more code to decipher the best program for a given
situation.
%
This shifts the problem of writing code to reading new, unfamiliar code.
%
We evaluate the addition of examples to type directed program synthesis.
%
We measured 10 participants between two variants, each with 4 tasks.
%
We do not find a statistical significance, in general, with the addition of
examples.
%
We do note, that one task did have a significant difference in performance
between groups.
%
Further evaluation is needed to determine a reason for this.
\end{abstract}


% \ccsdesc[500]{Computer systems organization~Embedded systems}
% \ccsdesc[300]{Computer systems organization~Redundancy}
% \ccsdesc{Computer systems organization~Robotics}
% \ccsdesc[100]{Networks~Network reliability}

%%
%% Keywords. The author(s) should pick words that accurately describe
%% the work being presented. Separate the keywords with commas.
% \keywords{datasets, neural networks, gaze detection, text tagging}


%%
%% This command processes the author and affiliation and title
%% information and builds the first part of the formatted document.
\maketitle
\section{Introduction}
Writing code is difficult and often inadvisable.


A recent approach to this classic conundrum is \emph{program synthesis} which
takes a specification and generates a program that implements it---saving us
from ourselves by letting the computer do hard part.
%
A specification may be a collection of input/output pairs
\cite{Feser_Chaudhuri_Dillig_2015}, a type signature \cite{hoogle_plus_2020},
or even a combination of the two \cite{Osera_Zdancewic_2015}.
%
These specifications are intentionally vague; many different
programs might meet the same specification.
%
While this keeps specification simple, it requires the synthesizer to either guess,
or, more realistically, produce multiple candidate programs.

Sifting through a list of candidate programs can be time-consuming, so much so
that it might be faster to write the code yourself.
%
If we could speed up this process, without requiring more input from the
user, we could allow programmers to both write simple specifications and
efficiently find their desired program.

In this work, we evaluate whether automatically generated input-output examples
for candidate programs will speed up the selection process.
%
We asked users to complete a series of synthesis tasks using a modified
version of, \hoogleplus, a program synthesizer for the Haskell
\cite{hoogle_plus_2020}, that produces input/output examples.
%
We find that while input/output examples do not appear to provide consistent
speedups, there is likely a set of circumstances where the speedup is
dramatic.

\section{Method}

% --- COPIED FROM ABSTRACT ASSIGNMENT --- %
% Explain your study design, illustrate how the study's results will provide
% evidence for/against your hypothesis, and how the question, hypothesis, and
% study all line up. We encourage you to mirror/copy/adapt other researchers
% methods (e.g. by drawing from the class readings) whenever appropriate (and
% not when it isn't appropriate).

% There are three major points you should hit here.

% Study design:
% What are you going to do? Be detailed and precise.

% Ecological Validity:
% Why does your study answer your research question? Why
% does your evaluation address your hypothesis? Make sure your study, and the
% variables you're measuring, properly address the question you are asking.
\subsection{Measures}
For each participant, we measured:
\begin{itemize}
    \item Time-to-correct solution (per task)
    \item Count of incorrect selections (per task)
    \item Count and type of interpreter queries (across all tasks)
\end{itemize}

The first metric directly tests the hypothesis that examples improve
selection speed, but the result could be misleading if there was also a
change in selection accuracy.
%
Counting incorrect guesses ensures that any such change is captured.
%
Beyond assessing the effectiveness of examples, measuring how often and why
participants used the interactive interpreter provides insight into any
behavior changes that might occur.

\subsection{Trials}
We had two variants.
%
The control version of the tool only showed programs that matched a type signature.
%
The treatment version showed three input-output examples with each program
suggested by the synthesizer.

Each participant went through one training task and four trial tasks.
%
The task was to find the best implementation of a type signature and a
description of what the program should do.
%
Timing began as soon as they were handed a card containing the task, and ended
when they pointed out the correct solution.
%
There was one correct answer per task---and task 3 had no correct solution.
%
Each participant was in a single variant for the entirety of their tasks.

\begin{figure*}[t!]
    \centering
    \begin{subfigure}[t]{0.5\textwidth}
        \centering
        \includegraphics[width=\textwidth]{method/control-ui.png}
        \caption{Control treatment, no examples}
    \end{subfigure}%
    ~
    \begin{subfigure}[t]{0.5\textwidth}
        \centering
        \includegraphics[width=\textwidth]{method/treatment-ui.png}
        \caption{Experimental treatment, examples}
    \end{subfigure}
    \caption{Each participant received one of the two treatments for the duration of the study.}
\end{figure*}

\section{Results}
\begin{figure*}[ht]
  \centering
  \includegraphics[width=\textwidth]{results/task_points.png}
  \caption{
    Examples did not noticeably affect time-to-completion, except in Task 2
    where the effect is dramatic.
  }
  \label{fig:data-points}
\end{figure*}
\subsection{Time-to-correct solution}
Cumulatively, across the four tasks, we did not observe a significant change
in the time-to-correct solution.
%
A Wilcoxon Rank-Sum test, performed on the aggregate of per-task rankings,
yielded a confidence interval $p=0.14$.
%
Performing rank-sum of ranks is an especially conservative test, because it weighs
per-task rankings equally, regardless of the absolute times.

Although the results do not indicate a change across the board, our data shows a dramatic
speedup for the example group in Task 2 (see Figure~\ref{fig:data-points}).
%
This may indicate that certain tasks benefit from examples while others do not.
%
We discuss this more thoroughly in Section~\ref{sec:discussion}

\subsection{Incorrect selections}
When participants made an incorrect selection, they were allowed to keep
selecting until they found the correct program.
%
This happened 3 times in the \noexamples group and 5 times in the example group.
%
In the example group, a single participant made 3 incorrect guesses on a
single task.
%
This participant was also the only one who ranked their Haskell knowledge as a 1 out of 5.
%
Futhermore, the incorrect selections contained incorrect input/output pairs. 
%
With these anomalies in mind, we don't believe this study has determined a definitive answer
on the relationship between examples and error rates.

\subsection{Interpreter usage}
In the \noexamples group, participants made a median of 6 type queries,
executed 3 subexpressions, and ran 3 full candidates across all 4 tasks.
%
In the \examples group, however, participants made a median of 0 queries of any kind.
%
It's clear that, introducing examples reduced participants reliance on the
interpreter to gather information.

% Figures.

\section{Discussion}
\label{sec:discussion}

\subsection{Hope for examples}
Although the study failed to show that examples aided candidate selection in
the general case, it's possible that in specific case it makes a large difference.
%
A Wilcoxon Rank-Sum test on the data from Task 2 shows that the \example
group was faster with a confidence of $p=0.03$.
%
While this could just be noise, the correct solution for Task 2 was 10th in
the list while Tasks 1 and 2 had their solution in the first two spots.
%
Task 3 had no solution.
%
If examples speed up candidate assessment, we'd expect this effect to be
magnified by the number of candidates assessed.
%
Because users had to sift through 9 incorrect solutions to get to the correct
one, it makes sense that Task 2 showed the most dramatic effect.
%
A study that randomizes solution location rather than preserving the synthesizer's ordering
could test this hypothesis.

\section{Future Work}
\label{sec:future}

\subsection{Methodology changes}
After encountering only one participant with minimal Haskell knowledge it became
clear we needed to randomize the treatment for each task.
%
This would mitigate the effects of outliers on any one variant, distributing
their effect on each treatment.
%

\subsection{Future Directions}
This work touches on but does not seek to evaluate whether recognition of a
correct solution or recall of necessary parts of a correct solution are more
useful in this type-directed search setting.
%
In our \noexamples group, participants largely relied on the interpreter to tell
them what different parts of a candidate solution were doing.
%
In effect, they used the interpreter as a source of documentation.
%
So, it remains to be seen what kinds of tasks users accomplish faster with
documentation as an aid vs input-output examples.
%




%%
%% The acknowledgments section is defined using the "acks" environment
%% (and NOT an unnumbered section). This ensures the proper
%% identification of the section in the article metadata, and the
%% consistent spelling of the heading.
\begin{acks}
To Robert, for the bagels and explaining CMYK and color spaces.
\end{acks}

%%
%% The next two lines define the bibliography style to be used, and
%% the bibliography file.
\bibliographystyle{ACM-Reference-Format}
\bibliography{bib}

%%
%% If your work has an appendix, this is the place to put it.
\appendix

\end{document}
\endinput
%%
%% End of file `sample-sigchi.tex'.
