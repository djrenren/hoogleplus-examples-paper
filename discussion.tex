\section{Discussion}
\label{sec:discussion}

\subsection{Hope for examples}
Although the study failed to show that examples aided candidate selection in
the general case, it's possible that in specific case it makes a large difference.
%
A Wilcoxon Rank-Sum test on the data from Task 2 shows that the \example
group was faster with a confidence of $p=0.03$.
%
While this could just be noise, the correct solution for Task 2 was 10th in
the list while Tasks 1 and 2 had their solution in the first two spots.
%
Task 3 had no solution.
%
If examples speed up candidate assessment, we'd expect this effect to be
magnified by the number of candidates assessed.
%
Because users had to sift through 9 incorrect solutions to get to the correct
one, it makes sense that Task 2 showed the most dramatic effect.
%
A study that randomizes solution location rather than preserving the synthesizer's ordering
could test this hypothesis.